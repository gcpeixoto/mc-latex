% bloco de código formatado
\usepackage{listings}
\usepackage{xcolor}
 
 % define cores como triplas (r,g,b) 0 < r,g,b < 1.
\definecolor{codegreen}{rgb}{0,0.6,0}
\definecolor{codegray}{rgb}{0.5,0.5,0.5}
\definecolor{codepurple}{rgb}{0.58,0,0.82}
\definecolor{backcolour}{rgb}{0.86,0.93,0.96}
 
\lstdefinestyle{mycode}{
    backgroundcolor=\color{backcolour},   
    commentstyle=\color{codegreen},
    keywordstyle=\color{magenta},
    numberstyle=\tiny\color{codegray},
    stringstyle=\color{codepurple},
    basicstyle=\ttfamily\footnotesize,
    breakatwhitespace=false,         
    breaklines=true,                 
    captionpos=b,                    
    keepspaces=true,                 
    numbers=left,                    
    numbersep=5pt,                  
    showspaces=false,                
    showstringspaces=false,
    showtabs=false,                  
    tabsize=2
}
 
\lstset{style=mycode}


% links no arquivo
\usepackage{hyperref}
\hypersetup{
    colorlinks=true,
    linkcolor=blue,
    filecolor=magenta,      
    urlcolor=cyan,
}
 
\urlstyle{same}

% efeito de papel rasgado 

\usepackage{framed}
\usepackage{tikz}
\usepackage[margin=1cm]{geometry}% for screen preview
\usetikzlibrary{decorations.pathmorphing,calc,shadows.blur,shadings}
\pgfmathsetseed{1} % To have predictable results
% Define a background layer, in which the parchment shape is drawn
\pgfdeclarelayer{background}
\pgfsetlayers{background,main}

% This is the base for the fractal decoration. It takes a random point between the start and end, and
% raises it a random amount, thus transforming a segment into two, connected at that raised point
% This decoration can be applied again to each one of the resulting segments and so on, in a similar
% way of a Koch snowflake.
\pgfdeclaredecoration{irregular fractal line}{init}
{
  \state{init}[width=\pgfdecoratedinputsegmentremainingdistance]
  {
    \pgfpathlineto{\pgfpoint{random*\pgfdecoratedinputsegmentremainingdistance}{(random*\pgfdecorationsegmentamplitude-0.02)*\pgfdecoratedinputsegmentremainingdistance}}
    \pgfpathlineto{\pgfpoint{\pgfdecoratedinputsegmentremainingdistance}{0pt}}
  }
}

% define some styles
\tikzset{
   paper/.style={draw=black!10, blur shadow,
                 lower left=black!20, upper left=black!15, upper right=white, lower right=black!10},
   irregular border/.style={decoration={irregular fractal line, amplitude=0.2},
           decorate,
     },
   ragged border/.style={ decoration={random steps, segment length=7mm, amplitude=2mm},
           decorate,
   }
}

% Macro to draw the shape behind the text, when it fits completly in the
% page
\def\tornpaper#1{
\tikz{
  \node[inner sep=1em] (A) {#1};  % Draw the text of the node
  \begin{pgfonlayer}{background}  % Draw the shape behind
  \fill[paper] % recursively decorate the bottom border
        decorate[irregular border]{decorate{decorate{decorate{decorate[ragged border]{
        ($(A.south east) - (0, random*5mm)$) -- ($(A.south west) - (0, random*5mm)$)
        }}}}}
        -- (A.north west) -- (A.north east) -- cycle;
  \end{pgfonlayer}}
}